\documentclass[a4paper,12pt]{article}

%%%%%%%%%%%%%%%%%%%%%%%%%%%%%%%%%%%%%%%%%%%%%%%%
% Packages
%%%%%%%%%%%%%%%%%%%%%%%%%%%%%%%%%%%%%%%%%%%%%%%%

\usepackage[right=2.5cm, left=2.5cm, top=2.5cm, bottom=2.5cm]{geometry} 
\usepackage[portuguese]{babel}
\usepackage[T1]{fontenc}
\usepackage[utf8]{inputenc}
\usepackage{url}
\usepackage{hyperref}
\Urlmuskip=0mu  plus 10mu

% no indentation
%\usepackage{setspace}
%\setlength{\parindent}{0in}

\usepackage{graphicx} 
\usepackage{float}
\usepackage{xcolor}

\usepackage{mathtools}
\usepackage{amssymb, amsthm}
\usepackage{cancel}

% headers
\usepackage{fancyhdr}

%%%%%%%%%%%%%%%%%%%%%%%%%%%%%%%%%%%%%%%%%%%%%%%%
% Proper definitions
%%%%%%%%%%%%%%%%%%%%%%%%%%%%%%%%%%%%%%%%%%%%%%%%
\newcommand{\R}{\mathbb{R}}
\newcommand{\B}{\mathcal{B}}
\newcommand{\sur}{\mathcal{S}}

\newtheoremstyle{exer}{}{}{\color{blue}}{}{\color{blue}\bfseries}{}{ }{}
\theoremstyle{exer}
\newtheorem{exercise}{Exercício}

\theoremstyle{definition}
\newtheorem{solution}{Solução}
\newtheorem{definition}{Definição}

\theoremstyle{plain}
\newtheorem{remark}{Observação}



%%%%%%%%%%%%%%%%%%%%%%%%%%%%%%%%%%%%%%%%%%%%%%%%
% Header (and Footer)
%%%%%%%%%%%%%%%%%%%%%%%%%%%%%%%%%%%%%%%%%%%%%%%%

\pagestyle{fancy} 
\fancyhf{}

\lhead{\footnotesize CS: Lista 8}
\rhead{\footnotesize Prof. Asla e Mon. Lucas} 
\cfoot{\footnotesize \thepage} 


\begin{document}

%%%%%%%%%%%%%%%%%%%%%%%%%%%%%%%%%%%%%%%%%%%%%%%%
% Title section of the document
%%%%%%%%%%%%%%%%%%%%%%%%%%%%%%%%%%%%%%%%%%%%%%%%

\thispagestyle{empty} 

\begin{tabular*}{0.95\textwidth}{l @{\extracolsep{\fill}} r} 
    {\large \bf Curvas e Superfícies 2021.1} &  \\
    Escola de Matemática Aplicada, Fundação Getulio Vargas &  \\
    Professora Asla Medeiros e Sá &  \\ 
    Monitor Lucas Machado Moschen & Entrega 14/06/2021\\
    \hline \\
\end{tabular*} 
\vspace*{0.3cm} 

\begin{center}
	{\Large \bf Lista 8}
	\vspace{2mm}
\end{center}  
\vspace{0.4cm}

\begin{exercise}
    Estudo do cilindro:
    \begin{enumerate}
    \item[(a)] Escolha uma parametrização de parte de uma superfície cilíndrica regular.
    \item[(b)] Desenhar, em software gráfico, as seções normais para um ponto da imagem da
    parametrização escolhida. Observe as direções em que as curvaturas das cuvas
    definidas pela seção normal são máxima e mínima.
    \item[(c)] Defina uma aplicação normal de Gauss para a parametrização escolhida.
    \item[(d)] Calcule os coeficientes da primeira forma fundamental para um ponto da parametrização proposta.
    \item[(e)] Calcule a área coberta pela parametrização proposta por você.
    \item[(f)] Calcule os coeficientes da segunda forma fundamental para o mesmo ponto analisado anteriormente.
    \item[(g)] Calcule as curvaturas principais e as direções principais para os pontos escolhidos
    do cilindro.   
\end{enumerate}
\end{exercise}

\begin{solution}

\end{solution}

\begin{exercise}
    Para o paraboloide hiperbólico dado pela parametrização $$X(u, v) = (u, v,
    v^2 - u^2), (u, v) \in \R^2,$$ faça:
    \begin{enumerate}
        \item[(a)] Desenhar, em software gráfico, as seções normais para um
        ponto do cilindro. Observe as direções em que as curvaturas das cuvas
        definidas pela seção normal são máxima e mínima. 
        \item[(b)] Calcule os coeficientes da segunda forma fundamental para o
        mesmo ponto $q = (0, 0)$.  
        \item[(c)] Calcule as curvaturas principais, a curvatura gaussiana e a
        curvatura média para esse ponto.
    \end{enumerate}
\end{exercise}

\begin{solution}

\end{solution}

\begin{exercise}
    Para a sela de macaco dada pela parametrização 
    $$X(u, v) = (u, v,
    u^3- 3uv^2), (u, v) \in \R^2,$$ faça:
    \begin{enumerate}
        \item[(a)] Desenhar, em software gráfico, as seções normais para um ponto do cilindro.
        Observe as direções em que as curvaturas das cuvas definidas pela seção normal
        são máxima e mínima.
        \item[(b)] Calcule os coeficientes da segunda forma fundamental para o mesmo ponto $q =
        (0, 0)$.
        \item[(c)] Calcule as curvaturas principais, a curvatura gaussiana e a curvatura média para
        esse ponto.
    \end{enumerate}
\end{exercise}

\begin{solution}
    
\end{solution}

\begin{exercise}
    Mostrar que planos são superfícies totalmente umbílicas.
\end{exercise}

\begin{solution}
    
\end{solution}

\begin{exercise}
    Mostrar que esferas são superfícies totalmente umbílicas.
\end{exercise}

\begin{solution}
    
\end{solution}

\begin{exercise}
    Detalhar a demonstração da página 4 das notas de aula (feitas também na
    aula em vídeo) que mostra que a curvatura das curvas definidas pelas
    seções normais coincide com a segunda forma fundamental. Isto é:
    $$
    k_{\alpha}(0) = \langle \alpha'', N(p) \rangle = \langle - dN_p(w), w \rangle = II_p
    $$
\end{exercise}

\begin{solution}
    
\end{solution}


% \begin{thebibliography}{9}
%      \bibitem{pressley} 
%      Pressley, Andrew N. Elementary differential geometry. Springer Science \& Business Media, 2010.
% \end{thebibliography}

\end{document}