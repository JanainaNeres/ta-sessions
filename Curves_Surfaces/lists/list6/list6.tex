\documentclass[a4paper,12pt]{article}

%%%%%%%%%%%%%%%%%%%%%%%%%%%%%%%%%%%%%%%%%%%%%%%%
% Packages
%%%%%%%%%%%%%%%%%%%%%%%%%%%%%%%%%%%%%%%%%%%%%%%%

\usepackage[right=2.5cm, left=2.5cm, top=2.5cm, bottom=2.5cm]{geometry} 
\usepackage[portuguese]{babel}
\usepackage[T1]{fontenc}
\usepackage[utf8]{inputenc}
\usepackage{url}
\usepackage{hyperref}
\Urlmuskip=0mu  plus 10mu

% no indentation
%\usepackage{setspace}
%\setlength{\parindent}{0in}

\usepackage{graphicx} 
\usepackage{float}
\usepackage{xcolor}

\usepackage{mathtools}
\usepackage{amssymb, amsthm}

% headers
\usepackage{fancyhdr}

%%%%%%%%%%%%%%%%%%%%%%%%%%%%%%%%%%%%%%%%%%%%%%%%
% Proper definitions
%%%%%%%%%%%%%%%%%%%%%%%%%%%%%%%%%%%%%%%%%%%%%%%%
\newcommand{\R}{\mathbb{R}}
\newcommand{\B}{\mathcal{B}}

\newtheoremstyle{exer}{}{}{\color{blue}}{}{\color{blue}\bfseries}{}{ }{}
\theoremstyle{exer}
\newtheorem{exercise}{Exercício}

\theoremstyle{definition}
\newtheorem{solution}{Solução}
\newtheorem{definition}{Definição}

\theoremstyle{plain}
\newtheorem{remark}{Observação}



%%%%%%%%%%%%%%%%%%%%%%%%%%%%%%%%%%%%%%%%%%%%%%%%
% Header (and Footer)
%%%%%%%%%%%%%%%%%%%%%%%%%%%%%%%%%%%%%%%%%%%%%%%%

\pagestyle{fancy} 
\fancyhf{}

\lhead{\footnotesize CS: Lista 6}
\rhead{\footnotesize Prof. Asla e Mon. Lucas} 
\cfoot{\footnotesize \thepage} 


\begin{document}

%%%%%%%%%%%%%%%%%%%%%%%%%%%%%%%%%%%%%%%%%%%%%%%%
% Title section of the document
%%%%%%%%%%%%%%%%%%%%%%%%%%%%%%%%%%%%%%%%%%%%%%%%

\thispagestyle{empty} 

\begin{tabular*}{0.95\textwidth}{l @{\extracolsep{\fill}} r} 
    {\large \bf Curvas e Superfícies 2021.1} &  \\
    Escola de Matemática Aplicada, Fundação Getulio Vargas &  \\
    Professora Asla Medeiros e Sá &  \\ 
    Monitor Lucas Machado Moschen & Entrega 26/05/2021\\
    \hline \\
\end{tabular*} 
\vspace*{0.3cm} 

\begin{center}
	{\Large \bf Lista 6}
	\vspace{2mm}
\end{center}  
\vspace{0.4cm}

\begin{exercise}
    Provar que toda bola aberta $\B(x; r)$ é um conjunto aberto.
\end{exercise}

\begin{solution}
    Seja $y \in \B(x; r)$. Queremos provar que existe $\epsilon > 0$ tal que
    $\B(y; \epsilon) \subseteq \B(r; x)$. Definimos para isto $\epsilon := r -
    |y - x| > 0$. Logo, dado qualquer ponto $z \in \B(y; \epsilon)$, temos que
    $$
    |z - x| \le |z - y| + |y - x| < \epsilon + |y - x| = r - |y - x| + |y - x| = r.
    $$
    Logo $z \in \B(x; r)$. Isto é, $\B(y; \epsilon) \subseteq \B(x; r)$. Concluímos que $\B(x; r)$ é aberto.
\end{solution}

\begin{exercise}
    Provar que $Z := \{(x, y) \in \R^2 : xy < 0\}$ é aberto. Dica: Seja $(a, b)$ no conjunto $Z$. Seja
    $\epsilon := \min\{|a|, |b|\} > 0$. Provar que $\B((a, b); \epsilon) \subseteq Z$.
\end{exercise}

\begin{solution}
    Usando a dica, tome $(x_1, x_2) \in \B((a,b), \epsilon)$, isto é,
    $$||(x_1 - a, x_2 - b)|| < \epsilon \implies (x_1 - a)^2 < \epsilon^2
    \text{ e } (x_2 - b)^2 < \epsilon^2,$$
    que, por seguinte, implica
    $$
    |x_1 - a| < \epsilon \text{ e } |x_2 - b| < \epsilon \implies x_1 \in (a - \epsilon, a + \epsilon), x_2 \in (b - \epsilon, b + \epsilon) 
    $$
    Se $a < 0$, então $a + \epsilon \le 0 \implies x_1 < 0$. Se $a > 0$, então
    $a - \epsilon \ge 0 \implies x_1 > 0$. O mesmo vale para $b$ e $x_2$.
    Portanto $\text{sign}(x_1x_2) = \text{sign}(ab) = -1$ o que implica $(x_1,
    x_2) \in Z$. 
\end{solution}

\begin{exercise}
    Provar que união de conjuntos abertos é um conjunto aberto.
\end{exercise}

\begin{solution}
    Seja $\{A_{\lambda} : \lambda \in \Lambda\}$ uma família de abertos, onde
    $\Lambda$ é um conjunto de índices (possívelmente infinito, não
    enumerável). Consideremos a união:
    $$
    A := \bigcup_{\lambda \in \Lambda} A_{\lambda}.
    $$
    Seja $z \in A$. Logo $z \in A_{\lambda}$ para algum índice $\lambda$. Dado
    que $A_{\lambda}$ é aberto, existe $\epsilon > 0$ tal que $\B(z; \epsilon)
    \subseteq A_{\lambda}$. Logo $\B(z; \epsilon) \subseteq A$. Concluímos que $A$ é aberto.
\end{solution}

\begin{exercise}
    Provar que a interseção de uma quantidade finita de abertos é um conjunto aberto.
\end{exercise}

\begin{solution}
    Seja $A = \bigcap_{i=1}^n A_i$ interseção de uma quantidade finita de
    abertos. Tome $z \in A$. Logo $z \in A_i$ para todo $i=1,...,n$. Dado que $A_i$ é aberto, existe $\epsilon_i > 0$ tal que $\B(z; \epsilon_i)
    \subseteq A_i$ (para todo $i=1,...,n$). Tome $\epsilon = \min_{i=1,...,n}
    \epsilon_i$. Assim, temos que, para todo $i$, 
    $$
    \B(z, \epsilon) \subseteq A_i
    $$
    e, portanto, $\B(z, \epsilon) \subseteq A$, o que prova que $A$ é aberto.
\end{solution}

\begin{exercise}
    Provar que a interseção de conjuntos fechados é um conjunto fechado. Será que união de fechados é também fechado? Se não for certo, dar um contraexemplo.
\end{exercise}

\begin{solution}
    Seja $\{A_{\lambda} : \lambda \in \Lambda\}$ uma família de fechados, onde
    $\Lambda$ é um conjunto de índices (possívelmente infinito, não
    enumerável). Consideremos a interseção:
    $$
    A := \bigcap_{\lambda \in \Lambda} A_{\lambda}.
    $$
    Veja que, pela lei de De Morgan,
    $$
    A^c = \bigcup_{\lambda \in \Lambda} A_{\lambda}^c
    $$
    é união de abertos (lembre que $A$ é fechado se, e somente se, $A^c$ é
    fechado) e, pelo exercício 3, $A^c$ é aberto. Concluímos que $A$ é
    fechado. De forma equivalente, utilizando o exercício 4, provamos que a
    união finita de fechados é fechada. Defina $A_n = \left[\frac{1}{n}, 1\right] \subset \R$ um intervalo fechado
    na reta. Assim 
    $$
    \bigcup_{n=1}^{\infty} A_n = (0,1]
    $$
    que não é fechado. 
\end{solution}

\begin{exercise}
    Dê exemplos de conjuntos que não são nem abertos nem fechados.
\end{exercise}

\begin{solution}
    \begin{itemize}
        Seguem alguns
        \item Conjuntos semi-abertos na reta, isto é, $(a,b]$ e
        $[a,b)$, com $a, b \in \R$. 
        \item $A = \{\frac{1}{n} : n \in \mathbb{N}\}$.
        \item União de discos fechados e abertos disjuntos.
    \end{itemize}
\end{solution}

\begin{exercise}
    Prove que 
    $$C = \{(x, y) \in \R^2 : y > 0\}$$
    é aberto.
\end{exercise}

\begin{solution}
    Tome $(a,b) \in C$ e $(z_1, z_2) \in \B((a,b), b)$. Assim
    $$
    ||(z_1 - a, z_2 - b)|| < b
    $$
    e, portanto, $|z_2 - b| < b$, isto é, $-b < z_2 - b < b \implies 0 < z_2 <
    2b$. Desta forma $(z_1, z_2) \in C$ o que implica $C$ ser aberto. 
\end{solution}

\begin{exercise}
    Prove que um conjunto em $\R^n$ é aberto se, e somente se, é união de bolas abertas.
\end{exercise}

\begin{solution}
    Pelo exercício 1 e 3, já está provado que a união de bolas abertas é um
    conjunto aberto. Tome $A \subseteq \R^n$ aberto. Para cada $x \in A$,
    existe $\epsilon_x > 0$ tal que $\B(x, \epsilon_x) \subseteq A$. Portanto, 
    $$
    \bigcup_{x \in A} \B(x, \epsilon_x) \subseteq A. 
    $$
    Além disso, $\forall x \in A, x \in \B(x,\epsilon_x)$
    $$
    A \subseteq \bigcup_{x \in A} \B(x, \epsilon_x)
    $$
    o que implica a igualdade. 
\end{solution}

\begin{exercise}
    Provar que $\R \times \{0\}$ é fechado em $\R^2$.
\end{exercise}

\begin{solution}
    Defina $A = (\R \times \{0\})^c = \R \times \{0\}^c$.
    Tome $(x,y) \in A$ e $(z_1, z_2) \in \B((x,y), |y|)$. Assim, como já vimos no
    exercício 7, 
    $$
    |z_2 - y| < |y| 
    $$
    e, portanto, $y - |y| < z_2 < y + |y|$. Se $y < 0$, $y + |y| = 0 \implies
    z_2 < 0$. Se $y > 0$, $y - |y|  = 0 \implies z_2 > 0$. Em ambos os casos,
    $z_2 \neq 0$, e, portanto, $(z_1, z_2) \in A$. Isso prova que $A$ é aberto
    e, por conseguinte, 
    $$
    A^c = \R \times \{0\}
    $$
    é fechado. 
\end{solution}

\begin{exercise}
    Prove que as bolas fechadas são conjuntos fechados.
\end{exercise}

\begin{solution}
    Seja $B = \bar{\B}(x,r)$ uma bola fechada com raio $r$. Tome $z \in B^c$,
    isto é 
    $$
    ||x - z|| > r.
    $$
    Tome $\epsilon = ||x - z|| - r > 0$. Vou provar que $\B(z,\epsilon)
    \subseteq B^c$. Seja $y \in \B(z,\epsilon)$. Portanto 
    $$
    ||y - z|| < \epsilon = ||x - z|| - r \implies r < ||x - z|| - ||y - z|| \le ||x - y||.
    $$
    Assim $y \in B^c$. Com isso demonstrado, provamos que $\B(z,\epsilon)
    \subseteq B^c$ e $B^c$ é aberto. Isso implica que $B$ é fechado. 
\end{solution}

\begin{exercise}
    Seja $A \subseteq \R^n$ tal que existe $d > 0$ tal que $||x - y|| \ge d$
    para todo par de pontos $x, y \in A$. Prove que $A$ é fechado em $\R^n$.
\end{exercise}

\begin{solution}
    Seja $x \in A^c$ e $B = \B(x,d/2)$. Suponha que exista $y \in B \cap A$.
    Dessa forma, para todo $z \in B$, 
    $$
    ||y-z|| \le ||y - x|| + ||x - z|| < \frac{d}{2} + \frac{d}{2} = d
    $$
    o que implica que se $z \neq y$, então $z \not \in A$. Isto é, $y$ é o
    único ponto que pertence a $B \cap A$. Tome $\epsilon = ||x - y|| < d/2$.
    Temos, então, que $y \not \in \B(x, \epsilon)$ e $\B(x,\epsilon) \subseteq
    \B(x, d/2) / \{y\} \subseteq A^c$. Provamos que para todo $x \in A^c$,
    existe $0 < \epsilon < d/2$ de forma que $\B(x,\epsilon) \subseteq A^c$, o
  que prova que $A^c$ é aberto e $A$ é fechado.   
\end{solution}

\begin{exercise}
    Seja $A \subseteq \R^2$ um conjunto não vazio contido numa reta de $\R^2$. Prove que $A$ não é
    aberto.
\end{exercise}

\begin{solution}
    Sejam $(u_1, u_2), (v_1, v_2) \in \R^2$ tal que $(v_1, v_2) \neq 0$ e de forma que 
    $$A \subseteq \{(u_1 + tv_1, u_2 + tv_2) | t \in \R\}$$
    Tome $x = (u_1 + tv_1, u_2 + tv_2) \in A$, $\epsilon > 0$ e $y =
    x + \alpha(-v_2,v_1)$, em que 
    $$0 < |\alpha| < \frac{\epsilon}{||v||}.$$ Assim  
    $$
    ||x - y|| = ||\alpha(-v_2, v_1)|| = |\alpha|||v|| < \epsilon,
    $$
    isto é, $y \in \B(x, \epsilon)$. Porém o sistema 
    $$
    \begin{cases}
        t v_1 - \alpha v_2 = s v_1  \\
        t v_2 + \alpha v_1 = s v_2
    \end{cases} \implies
    \begin{cases}
        (t - s)v_1 = \alpha v_2 \\
        (t - s)v_2 = -\alpha v_1
    \end{cases} \implies 
    -\frac{(t - s)^2}{\alpha}v_2 = \alpha
    v_2
    $$
    não tem solução em $s$, pois, se $v_2 \neq
    0$, então $-(t-s)^2 = \alpha^2$ que não tem solução em $s$ e, se $v_2 = 0$, temos,
    pela segunda equação $v_1 = 0$, o que é um absurdo. Isso implica que $y$
    não está na reta e, portanto, não está em $A$. Isso prova que $A$ não é
    aberto. 
\end{solution}

\begin{exercise}
    Seja $A \subseteq \R^n$. Prove que $\R^n/int(A)$ é fechado.
\end{exercise}

\begin{solution}
    Pela definição de interior de $A$, temos que $int(A)$ é aberto. Por esse
    motivo, $int(A)^c = \R^n / int(A)$ é fechado. 
\end{solution}

\begin{exercise}
    Seja $A \subset B \subseteq \R^n$, e $x$ ponto de acumulação de $A$. Será que $x$ é também ponto de
    acumulação de $B$?
\end{exercise}

\begin{solution}
    Se $x$ é ponto de acumulação de $A$, então existe uma sequência
    $\{x_n\}_{n \in \mathbb{N}} \subseteq A / \{x\}$ de forma que $x_n \to x$.
    De fato $\{x_n\}_{n \in \mathbb{N}} \subseteq B / \{x\}$, o que mostra que
    $x$ é ponto de acumulação de $B$. 
\end{solution}

\begin{exercise}
    Se $A \subset \R^n$ é aberto, prove que sua fronteira tem interior vazio.
\end{exercise}

\begin{solution}
    Denote $\partial A$ os pontos da fronteira de $A$. Se $x \in \partial A$,
    então para todo $\epsilon > 0$,
    $$
    \B(x,\epsilon) \cap A \neq \emptyset \text{ e } \B(x,\epsilon) \cap A^c \neq \emptyset .
    $$
    Suponha que existe $x \in int(\partial A)$. Assim, existe $\epsilon > 0$
    tal que $\B(x, \epsilon) \subseteq \partial A$,     $\B(x,\epsilon)
    \cap A \neq \emptyset$ e $\B(x,\epsilon)
    \cap A^c \neq \emptyset$. Tome $y \in \B(x, \epsilon) \cap A$. Como $A$ é
    aberto, existe $r > 0$ tal que $\B(y,r) \subseteq A$, logo $\B(y, r) \cap A^c = \emptyset$,  porém $y \in
    \partial A$, o que é um absurdo. Portanto $int(\partial A) = \emptyset$. 
\end{solution}

\begin{exercise}
    Seja $A \subseteq \R^n$ com $n \ge 2$. Prove que, dado $a \in \R^n/A$, o
    conjunto $A \cup \{a\}$ é aberto se, e somente se, $a$ é um ponto isolado da
    fronteira de $A$. 
\end{exercise}

\begin{solution}
    Separamos a ida e a volta
    \begin{enumerate}
        \item[$\Rightarrow$] Suponha que $A \cup \{a\}$ seja aberto. Primeiro
        vamos provar que $a \in \partial A$. Como $a \in A^c$, temos que para
        todo $\epsilon > 0$, $\B(a, \epsilon) \cap A^c \neq 0$. Além disso,
        como $A \cup \{a\}$ é aberto, existe $\alpha > 0$ tal que
        $\B(a,\alpha)/\{a\} \subseteq A$. Desta forma, para todo $\epsilon >
        0$, $\B(a, \epsilon) \cap A \neq 0$. Provamos que $a \in \partial A$. 
        
        Preciso provar que $a$ é ponto isolado de $\partial A$, isto é,
        existe $r > 0$ de forma que
        $$\B(a,r) \cap \partial A = \{a\}.$$ 
        Seja $r = \alpha/2$. Suponha que exista $y \in \B(a, r) \cap \partial
        A$ diferente de $a$. Assim, para todo $\epsilon > 0$, temos que
        $\B(y,\epsilon) \cap A \neq \emptyset$ e $\B(y, \epsilon) \cap A^c
        \neq \emptyset$. Tome $\epsilon = ||y - a||>0$ e $z \in
        \B(y,\epsilon)$. Assim:
        $$
        ||z - a|| \le ||z - y|| + ||y - a|| < 2r = \alpha.
        $$
        Além disso,
        $$
        ||z - a|| \ge ||y-a|| - ||y-z|| > ||y - a|| - ||y-a|| = 0.
        $$
        Portanto $z \in \B(a,\alpha)/\{a\} \subseteq A$ o que implica
        $\B(y,\epsilon) \subseteq A$, o que é um absurdo. Portanto $a$ é ponto
        isolado de $\partial A$.
        \item[$\Leftarrow$] Seja $A = \{x \subseteq \R^n : 0 < ||x|| < 1\}$ e
        $a = 0$. Portanto $a$ é um ponto isolado de $\partial A = \{x \in \R^n
        |\, x = 0 \text{ ou } ||x|| = 1\}$, mas $A \cup \{a\} = \{x \in \R^n |
        1 \ge ||x||\}$ é fechado, o que é um contra exemplo. 
    \end{enumerate}
\end{solution}

\begin{exercise}
    Prove que se $F \subseteq \R^n$ é fechado então sua fronteira tem interior vazio.
\end{exercise}

\begin{solution}
    Suponha que exista $x \in int(\partial F)$. Então existe $\epsilon > 0$
    tal que $\B(x,\epsilon) \subseteq \partial F$. Tome $y \in \B(x,\epsilon)
    \cap F^c$. Como $F^c$ é aberto, porque $F$ é fechado, existe $r > 0$ dal
    que $\B(y,r) \subseteq F^c$. Porém $y \in \partial F$ e, portanto,
    $\B(y,r) \cap F \neq \emptyset$, o que é um absurdo. Portanto
    $int(\partial F) = \emptyset$. 
\end{solution}

\begin{exercise}
    Sejam $F \in \R^n$ fechado e $f : F \to \R^m$ uma aplicação contínua.
    Mostre que $f$ leva subconjuntos limitados de $F$ em subconjuntos
    limitados de $\R^m$. Prove, exibindo um contra-exemplo, que não se conclui o mesmo removendo-se a hipótese de $F$ ser fechado.
\end{exercise}

\begin{solution}
    Seja $A \subset F$ limitado e defina $B = F(A) \subseteq \R^m$. Vamos
    provar esse exercício com um pouco de análise. Suponha que $B$ não é
    limitado, isto é, para todo $r > 0$, existe $y \in B$ de forma que $||y||
    > r$. Assim existe uma sequência $\{a_n\}_{n \in \mathbb{N}} \subset
    A$ de forma que $||f(a_n)|| \ge n$. Como $A$ é limitado, temos que a
    sequência $\{a_n\}$ é limitada. Pelo Teorema de Bolzano-Weierstrass,
    existe uma subsequência $\{a_{n_k}\}$ convergente para $a^*$. Como $F$ é fechado, todos os seus pontos de aderência pertencem ao
    conjunto, e, portanto, $a^* \in F$. Assim $a_{n_k} \to a^*$ e, pela
    continuidade de $f$, temos que $f(a_{n_k}) \to f(a^*)$, o que é um
    absurdo, dado que $||f(n_k)||$ diverge. Concluímos que $B$ é limitado.
\end{solution}

\begin{remark}
    Essa solução se usou de Teorema de Bolzano-Weierstrass, que $F =
    \bar{F}$, quer dizer, se $a$ é limite de uma sequência de pontos de $F$,
    então $a \in F$ e continuidade de $f$ através de sequências. Uma
    demonstração de topologia é bem-vinda. 
\end{remark}

\begin{remark}
    Seja $f : (0,1] \to \R$ de foram que $f(x) = 1/x$. Temos que $f(0,1] = [1,
    + \infty)$ que não é limitado. Isso acontece justamente porque o limite da
    subsequência pode não pertencer a $F$, se ele não é fechado. Desta forma
    $f(a^*)$ não estará definido. 
\end{remark}

\begin{exercise}
    Prove que duas bolas abertas de $\R^n$ são homeomorfas.
\end{exercise}

\begin{solution}
    Dados $a \in \R^n$ e $r > 0$, consideremos a aplicação:
    \begin{align*}
        f : \B(0, 1) &\to \B(a, r)\\
        x &\mapsto rx + a        
    \end{align*}
    A aplicação $f$ é bijetiva e contínua. Sua inversa, $f^{-1} : \B(a, r) \to
    \B(0, 1)$, é dada por $f^{-1}(y) = \frac{1}{r}(y - a)$, donde se vê que
    $f^{-1}$ é contínua, portanto $f$ é um homeomorfismo. Pela transitividade
    da relação de homeomorfismo, conclui-se que duas bolas abertas quaisquer de
    $\R^n$ são homeomorfas. Um argumento análogo prova que vale o mesmo para duas bolas, ambas, fechadas.
\end{solution}

\begin{exercise}
    Verifique que a aplicação:
    \begin{align*}
        f : \B(0, 1) &\to \R^n \\
        x &\mapsto \frac{x}{1 - ||x||}        
    \end{align*}
    é um homeomorfismo entre a bola aberta unitária $\B(0, 1)$ e $\R^n$.
    Conclua que qualquer bola aberta de $\R^n$ é homeomorfa a todo o espaço $\R^n$.
\end{exercise}

\begin{solution}
    Defina 
    \begin{align*}
        g : \R^n &\to \B(0,1) \\
        y&\mapsto \frac{y}{1 + ||y||} 
    \end{align*}
    Primeiro veja que $||g(y)|| < 1$ para todo $y$ e, portanto, $g$ está
    definida. Vejamos que $g = f^{-1}$. Tome $x \in \B(0,1)$. Então
    $$
    g(f(x)) = g\left(\frac{x}{1-||x||}\right) = \frac{\frac{x}{1 - ||x||}}{1 + \frac{||x||}{1 - ||x||}} = x
    $$
    e se $y \in \R^n$,
    $$
    f(g(y)) = f\left(\frac{y}{1 + ||y||}\right) = \frac{\frac{y}{1+||y||}}{1-\frac{||y||}{1+||y||}} = y
    $$
    o que demonstra que $g = f^{-1}$. Como $f$ possui inversa, é bijetiva. 

    Vamos demonstrar a continuidade de $f$. 
    Considere $A \subseteq \R^n$ aberto e $B = f^{-1}(A)$. Vamos mostrar que
    $B$ é aberto, isto é, para todo $x \in B$,
    existe $r > 0$ tal que $\B(x, r) \subset B$. Tome $x = f^{-1}(y) \in B$.
    Como $A$ é aberto, existe $\epsilon > 0$ tal que $\B(y, \epsilon)
    \subset A$. Tome $\delta$ tal que 
    $$
    \frac{\delta}{1 - ||x|| - \delta}(1 + ||y||) < \epsilon
    $$
    e $z = f^{-1}(w) \in \B(x,\delta)$. 
    \begin{equation*}
        \begin{split}
            ||y - w|| &= \left|\left|\frac{x}{1 - ||x||} - \frac{z}{1 - ||z||}\right|\right| = \frac{1}{1 - ||x||}\left|\left|x - \frac{1 - ||x||}{1 - ||z||}z\right|\right| \\ 
            &= \frac{1}{1 - ||x||}\left|\left|x - z + z - \frac{1 - ||x||}{1 - ||z||}z\right|\right| \\
            &\le \frac{||x-z||}{1 - ||x||} + \frac{1}{1 - ||x||}\left(1 - \frac{1 - ||x||}{1 - ||z||}||z||\right) \\
            &= \frac{||x-z||}{1 - ||x||} + \frac{||z||}{1 - ||x||}\frac{||x|| - ||z||}{1 - ||z||} \\
            &\le \frac{1}{1 - ||x||}||x-z||(1 + ||w||) \\
            &\le \frac{1}{1 - ||x||}||x-z||(1 + ||y - w|| + ||y||) \\
            \implies ||y - w|| &\le \frac{||x-z||}{1 - ||x|| - ||x-z||}(1 + ||y||) \\ 
            &< \frac{\delta}{1 - ||x|| - \delta}(1 + ||y||) < \epsilon
        \end{split}
    \end{equation*}
    Então $w \in \B(y, \epsilon) \subset A \implies z \in B$, o que prova que
    $B$ é aberto. Isso conclui a continuidade de $f$. A continuidade de
    $f^{-1}$ tem demonstração similar. Isso prova que $f$ é homeomorfismo e,
    com o exercício 19 mais a transitividade da relação "homeomorfa a", temos
    que qualquer bola aberta em $\R^n$ é homeomorfa a $\R^n$. 
\end{solution}

\begin{exercise}
    Mostre que o cone $C = \{(x, y, z) \in \R^3 ; z = x^2 + y^2 \}$ e $\R^2$ são homeomorfos.
\end{exercise}

\begin{solution}
    Observe que isso é similar a argumentar que o cone é uma superfície.
    Defina 
    \begin{align*}
        \sigma : \R^2 &\to C \\
        (x,y) &\mapsto (x,y,x^2+y^2).
    \end{align*}
    Primeiro vou provar que $\sigma$ possui inversa. Defina
    \begin{align*}
        \sigma^{-1} : C &\to \R^2 \\
        (x,y,z) &\mapsto (x,y)
    \end{align*}
    Tome $(x,y) \in \R^2$. Assim $\sigma^{-1}(\sigma(x,y)) =
    \sigma^{-1}(x,y,x^2+y^2)=(x,y)$. Tome $(x,y,z) \in C$, isto é, $z = x^2 +
    y^2$. Assim $\sigma(\sigma^{-1}(x,y,z)) = \sigma(x,y) = (x,y,x^2+y^2)=
    (x,y,z)$.  Provamos que $\sigma$ possui inversa e, portanto, é bijetiva. A
    continuidade de $\sigma$ e $\sigma^{-1}$ vem da continuidade de cada
    componente. Portanto $\sigma$ é um homeomorfismo o que prova que $\R^2$ e
    $C$ são homeomorfos.
\end{solution}

% \begin{thebibliography}{9}
%     \bibitem{pressley} 
%     Pressley, Andrew N. Elementary differential geometry. Springer Science \& Business Media, 2010.
% \end{thebibliography}

\end{document}