\documentclass[a4paper,12pt]{article}

%%%%%%%%%%%%%%%%%%%%%%%%%%%%%%%%%%%%%%%%%%%%%%%%
% Packages
%%%%%%%%%%%%%%%%%%%%%%%%%%%%%%%%%%%%%%%%%%%%%%%%

\usepackage[right=2.5cm, left=2.5cm, top=2.5cm, bottom=2.5cm]{geometry} 
\usepackage[portuguese]{babel}
\usepackage[T1]{fontenc}
\usepackage[utf8]{inputenc}
\usepackage{url}
\usepackage{hyperref}
\Urlmuskip=0mu  plus 10mu

% no indentation
%\usepackage{setspace}
%\setlength{\parindent}{0in}

\usepackage{graphicx} 
\usepackage{float}
\usepackage{xcolor}

\usepackage{mathtools}
\usepackage{amssymb, amsthm}

% headers
\usepackage{fancyhdr}

%%%%%%%%%%%%%%%%%%%%%%%%%%%%%%%%%%%%%%%%%%%%%%%%
% Proper definitions
%%%%%%%%%%%%%%%%%%%%%%%%%%%%%%%%%%%%%%%%%%%%%%%%
\newcommand{\R}{\mathbb{R}}

\newtheoremstyle{exer}{}{}{\color{blue}}{}{\color{blue}\bfseries}{}{ }{}
\theoremstyle{exer}
\newtheorem{exercise}{Exercício}

\theoremstyle{definition}
\newtheorem{solution}{Solução}

\theoremstyle{plain}
\newtheorem{remark}{Observação}



%%%%%%%%%%%%%%%%%%%%%%%%%%%%%%%%%%%%%%%%%%%%%%%%
% Header (and Footer)
%%%%%%%%%%%%%%%%%%%%%%%%%%%%%%%%%%%%%%%%%%%%%%%%

\pagestyle{fancy} 
\fancyhf{}

\lhead{\footnotesize CS: Lista 5}
\rhead{\footnotesize Prof. Asla e Mon. Lucas} 
\cfoot{\footnotesize \thepage} 


\begin{document}

%%%%%%%%%%%%%%%%%%%%%%%%%%%%%%%%%%%%%%%%%%%%%%%%
% Title section of the document
%%%%%%%%%%%%%%%%%%%%%%%%%%%%%%%%%%%%%%%%%%%%%%%%

\thispagestyle{empty} 

\begin{tabular*}{0.95\textwidth}{l @{\extracolsep{\fill}} r} 
    {\large \bf Curvas e Superfícies 2021.1} &  \\
    Escola de Matemática Aplicada, Fundação Getulio Vargas &  \\
    Professora Asla Medeiros e Sá &  \\ 
    Monitor Lucas Machado Moschen & Entrega 13/05/2021\\
    \hline \\
\end{tabular*} 
\vspace*{0.3cm} 

\begin{center}
	{\Large \bf Lista 5}
	\vspace{2mm}
\end{center}  
\vspace{0.4cm}

\begin{exercise}
    Seja $F : \R^2 \to \R^3$ uma aplicação linear. Mostre que: $F$ é injetora
    se, e só se, a imagem da base canônica de $\R^2$ forma um conjunto de
    vetores linearmente independentes de $\R^3$ ou, equivalentemente, se a
    matriz associada de $F$ tem posto 2. (obs.: Repare que este resultado está
    sendo usado para o conceito de superfície regular descrito acima).
\end{exercise}

\begin{solution}

\end{solution}

\begin{exercise}
    Mostre que o paraboloide hiperbólico $S = \{(x, y, z) \in \R^3; z = x^2 -
    y^2\}$ é uma superfície regular. Desenhe o paraboloide em um ambiente
    gráfico juntamente com o plano tangente e um vetor normal à superfície.
    Faça o des enho de forma a poder variar o ponto aonde o plano tangente é exibido.
\end{exercise}

\begin{solution}
    
\end{solution}

\begin{exercise}
    Mostre que, se $f(u, v)$ é uma função real diferenciável, onde $(u, v) \in
    U$, aberto de $\R^2$, então a aplicação $X(u, v) = (u, v, f(u, v))$ é uma
    superfície parametrizada regular, que descreve o gráfico da função $f$.
\end{exercise}

\begin{solution}
    
\end{solution}

\begin{exercise}
    Considere o hiperbolóide de uma folha
    $$
    S := \{(x, y, z) \in \R^3 : x^2 + y^2 - z^2 = 1\}
    $$
    Mostre que, para todo $\theta$, a reta
    $$
    (x - z)\cos(\theta) = (1 - y)\sin(\theta), (x + z)\sin(\theta) = (1 + y)\cos(\theta)
    $$
    está contida em $S$, e que, todo ponto do hiperboloide está em alguma
    dessas linhas. Desenhe o hiperbolóide e as linhas em um ambiente gráfico.
    Deduza que a superfície pode ser coberta por uma única parametrização.
\end{exercise}

\begin{solution}
    
\end{solution}

\begin{exercise}
    Considere uma curva regular $\alpha(s) = (x(s), y(s), z(s)), s \in I
    \subset \R$. Seja o subconjunto de $\R^3$ gerado pelas retas que passam
    por $\alpha(s)$, paralelas ao eixo $O_z$. Dê uma condição suficiente que
    deve satisfazer a curva $\alpha$ para que $S$ seja o traço de uma
    superfície parametrizada regular.
\end{exercise}

\begin{solution}
    
\end{solution}

\begin{exercise}
    {\bf Extra:} Mostre que o cilindro circular
    $$
    S := \{(x, y, z) \in \R^3 : x^2 + y^2 = 1\}
    $$
    pode ser descrito por uma parametrização global, isto é, que existe um
    atlas composto só por uma única carta.
\end{exercise}

\begin{solution}
    \url{https://math.stackexchange.com/questions/1664320/showing-a-circular-cylinder-is-a-surface}
\end{solution}


% \begin{thebibliography}{9}
%     \bibitem{pressley} 
%     Pressley, Andrew N. Elementary differential geometry. Springer Science \& Business Media, 2010.
% \end{thebibliography}

\end{document}