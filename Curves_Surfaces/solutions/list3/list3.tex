\documentclass[a4paper,12pt]{article}

%%%%%%%%%%%%%%%%%%%%%%%%%%%%%%%%%%%%%%%%%%%%%%%%
% Packages
%%%%%%%%%%%%%%%%%%%%%%%%%%%%%%%%%%%%%%%%%%%%%%%%

\usepackage[right=2.5cm, left=2.5cm, top=2.5cm, bottom=2.5cm]{geometry} 
\usepackage[portuguese]{babel}
\usepackage[T1]{fontenc}
\usepackage[utf8]{inputenc}
\usepackage{url}
\usepackage{hyperref}
\Urlmuskip=0mu  plus 10mu

% no indentation
%\usepackage{setspace}
%\setlength{\parindent}{0in}

\usepackage{graphicx} 
\usepackage{float}
\usepackage{xcolor}

\usepackage{mathtools}
\usepackage{amssymb, amsthm}

% headers
\usepackage{fancyhdr}

%%%%%%%%%%%%%%%%%%%%%%%%%%%%%%%%%%%%%%%%%%%%%%%%
% Proper definitions
%%%%%%%%%%%%%%%%%%%%%%%%%%%%%%%%%%%%%%%%%%%%%%%%
\newcommand{\R}{\mathbb{R}}

\newtheoremstyle{exer}{}{}{\color{blue}}{}{\color{blue}\bfseries}{}{ }{}
\theoremstyle{exer}
\newtheorem{exercise}{Exercício}

\theoremstyle{definition}
\newtheorem{solution}{Solução}

\theoremstyle{plain}
\newtheorem{remark}{Observação}



%%%%%%%%%%%%%%%%%%%%%%%%%%%%%%%%%%%%%%%%%%%%%%%%
% Header (and Footer)
%%%%%%%%%%%%%%%%%%%%%%%%%%%%%%%%%%%%%%%%%%%%%%%%

\pagestyle{fancy} 
\fancyhf{}

\lhead{\footnotesize CS: Lista 3}
\rhead{\footnotesize Prof. Asla e Mon. Lucas} 
\cfoot{\footnotesize \thepage} 


\begin{document}

%%%%%%%%%%%%%%%%%%%%%%%%%%%%%%%%%%%%%%%%%%%%%%%%
% Title section of the document
%%%%%%%%%%%%%%%%%%%%%%%%%%%%%%%%%%%%%%%%%%%%%%%%

\thispagestyle{empty} 

\begin{tabular*}{0.95\textwidth}{l @{\extracolsep{\fill}} r}
    {\large \bf Curvas e Superfícies 2021.1} &  \\
    Escola de Matemática Aplicada, Fundação Getulio Vargas &  \\
    Professora Asla Medeiros e Sá &  \\ 
    Monitor Lucas Machado Moschen & Entrega 21/03/2022 \\
    \hline \\
\end{tabular*} 
\vspace*{0.3cm} 

\begin{center}
	{\Large \bf Lista 3} 
	\vspace{2mm}
\end{center}  
\vspace{0.4cm}

\begin{exercise}
    Verifique a regularidade e calcule o comprimento de arco e a curvatura das
    seguintes curvas, quando possível:
    \begin{enumerate}
        \item (retas) $\alpha(t) = (a + ct, b + dt), t \in \R$;
        \item $\alpha(t) = (t,t^4), t \in \R$;
        \item (círculos) $\alpha(s) = (a + r cos(r/s), b + r \sin(r/s)), s \in
        \R, r > 0$;
        \item (cardióide) $\alpha(t) = (\cos(t)(2\cos(t) - 1),
        \sin(t)(2\cos(t) - 1)), t \in \R)$; 
        \item (catenária) $\alpha(t) = (t, \cosh(t)), t \in \R$.
    \end{enumerate}
\end{exercise}

\begin{solution}
    Considere:
    \begin{enumerate}
        \item $\alpha '(t) = (c,d)$. Portanto $\alpha$ é regular se, e somente
        se, $c \neq 0$ ou $d \neq 0$. Como $\alpha$ é uma reta, isso está
        claro e, portanto, é regular. O seu comprimento de arco é dado por
        $$L_0^t(\alpha) = \int_0^t (c^2 + d^2)^{1/2} ds = (c^2 +
        d^2)^{1/2}t.$$
        Para calcular a curvatura, precisamos de $\alpha ''(t) = (0,0)$, o que
        implica 
        $$\kappa_{\alpha}(t) = \langle J(c,d), (0,0) \rangle = 0.$$

        \item $\alpha '(t) = (1,4t^3)$. Portanto $\alpha$ é regular dado que a
        primeira componente de sua derivada é nunca nula. O seu comprimento de arco é dado por
        $$L_0^t(\alpha) = \int_0^t (1 + 16s^6)^{1/2} ds$$
        Observe que essa curva não está parametrizada pelo comprimento de
        arco.  Para calcular a curvatura, precisamos de $\alpha ''(t) = (0,12t^2)$, o que
        implica 
        $$\kappa_{\alpha}(t) = \frac{1}{||\alpha '(t)||^3}\langle J(1,4t^3),
        (0,12t^2) \rangle = \frac{12t^2}{(1 + 16t^6)^{3/2   }}.$$

        Os próximos seguem a mesma lógica, lembrando que para curvatura é
        importante a divisão por $||\alpha '(t)||^3$ e quando a curva é
        parametrizada pelo comprimento de arco, esse valor será igual a 1.
    \end{enumerate}

\end{solution}

\begin{exercise}
    Considere a elipse $\beta(t) = (a\cos(t);b\sin(t)), t \in \R$, onde $a >
    0, b > 0$ e $a \neq b$. Obtenha os valores de $t$ onde a curvatura de
    $\beta$ é máxima e mínima.
\end{exercise}

\begin{solution}
    Primeiro, observe quw $\beta '(t) = (-a\sin(t), b\cos(t))$ e $\beta ''(t)
    = -\beta(t)$. Essa curva não é parametrizada pelo comprimento de
    arco e $||\beta '(t)|| = (a^2\sin(t)^2 + b^2\cos(t))^{1/2}$, reescrevendo
    $||\beta'(t)|| = (a^2 + (b^2 - a^2)\cos^2(t))^{1/2}$. Assim, 
    $$
    \kappa_{\beta}(t) = \frac{1}{ (a^2 + (b^2 - a^2)\cos^2(t))^{3/2}}\langle J\beta '(t), -\beta(t) \rangle = \frac{ab}{(a^2 + (b^2 - a^2)\cos^2(t))^{3/2}}$$
    Podemos ver que a expressão $a^2 + (b^2 - a^2)\cos^2(t)$ está entre os
    extremos $a^2$ e $b^2$, dependendo se $a > b$ ou o contrário. 

    Assim, se $b > a$, a curvatura será máxima quando essa expressão for
    mínima, isto é, quando $cos(t) = 0 \implies t = \pm \pi/2$ e será máxima
    quando $cos(t) = \pm 1 \implies t = 0$ ou $t = \pi$. Se $a > b$, os
    valores são trocados para mínimo e máximo. 
\end{solution}

\begin{exercise}
    Seja $I = (-a,a), a > 0$ um intervalo aberto em $\R$, o qual é simétrico
    com respeito à origem. Seja $\alpha : I \to \R^2$ uma curva regular
    parametrizada por comprimento de arco. 
    \begin{itemize}
        \item Mostre que $\beta : I \to \R^2$, em que $\beta(s) = \alpha(-s)$,
        é uma curva regular, parametrizada pelo comprimento de arco e que
        satisfaz $\kappa_{\beta}(s) = -\kappa_{\alpha}(-s), \forall s \in I$.
        \item Desenhe em ambiente computacional um par de parametrizações de
        uma curva que ilustre o fato demonstrado no item anterior.
    \end{itemize}
\end{exercise}

\begin{solution}
    Primeiro observamos que $\beta '(s) = -\alpha(-s) \implies ||\beta'(s)|| =
    ||\alpha '(-s)|| = 1,$ $\forall s \in I$, o que implica que a curva é
    regular e parametrizada pelo comprimento de arco. Além disso $\beta''(s) =
    \alpha''(-s)$ e, portanto, 
    $$
    \kappa_{\beta}(s) = \langle J\beta '(s), \beta''(s) \rangle = - \langle J\alpha'(-s), \alpha ''(s) \rangle = - \kappa_{\alpha}(-s) 
    $$
    O desenho em GeoGebra pode ser encontrado no Github, na pasta usual da
    disciplina. 
\end{solution}

\begin{exercise}
     Reproduza, no ambiente computacional de sua preferência, os vetores $\alpha'$ e $\alpha''$ para uma curva $\alpha$ de sua preferência com duas parametrizações distintas, sendo uma a parametrização por comprimento de arco. Use como referência o arquivo \texttt{.gbb} mostrado em aula.
\end{exercise}

\begin{exercise}
    Reproduza, no ambiente computacional de sua preferência, a representação gráfica da
    \textit{Tangente Indicatrix}, conforme exemplificado no arquivo \texttt{.ggb} apresentado em aula. Teste com a curva de sua preferência.
\end{exercise}

\begin{exercise}
    Considerando o conceito de derivada como aproximação linear. Considere a
    aplicação: 
    \begin{gather}
        f: \R^2 \to \R^2 \\
        (x,y) \to (x^3 + y^3, x^3 - y^3) 
    \end{gather}
    determine suas derivadas $f'(x)$ e $f''(x)$. 
\end{exercise}

\begin{solution}
    $$
    f'(x,y) =  \begin{bmatrix}
        \frac{d}{dx}(x^3 + y^3) & \frac{d}{dy}(x^3 + y^3) \\
        \frac{d}{dx}(x^3 - y^3) & \frac{d}{dy}(x^3 - y^3) 
    \end{bmatrix} = \begin{bmatrix}
        3x^2 & 2y^2 \\
        3x^2 & -2y^3 
    \end{bmatrix} 
    $$
    Para a notação da segunda derivada, consulte \url{https://en.wikipedia.org/wiki/Hessian_matrix#Vector-valued_functions}.
    $$
    f''(x,y) = \begin{bmatrix}
        \begin{bmatrix}
            \frac{d^2}{dx^2}(x^3 + y^3) & \frac{d^2}{dxy}(x^3 + y^3) \\
            \frac{d^2}{dyx}(x^3 + y^3) & \frac{d^2}{dy^2}(x^3 + y^3)
        \end{bmatrix} & 
        \begin{bmatrix}
            \frac{d^2}{dx^2}(x^3 - y^3) & \frac{d^2}{dxy}(x^3 - y^3) \\
            \frac{d^2}{dyx}(x^3 - y^3) & \frac{d^2}{dy^2}(x^3 - y^3)
        \end{bmatrix}
    \end{bmatrix}
    $$
    Então 
    $$
    f''(x) = \begin{bmatrix}
        \begin{bmatrix}
            6x & 0 \\
            0 & 6y
        \end{bmatrix} & 
        \begin{bmatrix}
            6x & 0 \\
            0 & -6y
        \end{bmatrix}
    \end{bmatrix} 
    $$

\end{solution}

\begin{exercise}
    Uma aplicação $\Phi : \R^2 \to \R^2$ é dita {\it movimento rígido} quando
    preserva distâncias. Isto é: 
    $$
    ||\Phi(p) - \Phi(q)|| = ||p - q||
    $$
    Verifica-se que todo movimento rígido se escreve em forma única como
    composta de uma transformação linear ortogonal e uma translação, ou seja:
    $$
    \Phi(p) = Ap + p_0, \forall p \in \R^2,
    $$
    em que $A : \R^2 \to \R^2$  é um operador linear ortogonal e  $p_0$ um
    ponto de $\R^2$. Diz-se que $\Phi$ é {\it direto} ou {\it inverso},
    conforme $\det(A) = 1$ ou $-1$ respectivamente. Verifique que $\Phi$ é
    diferenciável e calcule $\Phi '(p)$ e $\Phi ''(p)$.
\end{exercise}

\begin{solution}
    Tome $a \in \R^2$ e defina $r_a(v) = \Phi(a + v) - \Phi(a) - Av$, tal que
    $v \in \R^2$. Eu gostaria de provar que 
    $$
    \lim_{v \to 0} \frac{r_a(v)}{||v||} = 0, \forall a \in \R^2
    $$
    Podemos reescrever $r_a(v) = A(a + v) + p_0 - (Aa + p_0) - Av = 0$, o que
    prova que $\Phi$ é diferenciável e $\Phi '(p) = A$. Observe que $\phi '$ é
    uma função constante e, portanto $\Phi ''(p) = 0$ em todas as suas
    componentes. Se pensamos que $\Phi ' : \R^2 \to L(\R^2, \R^2)$, $\Phi
    ''(p)$ será um tensor de ordem 3 com 8 entradas, todas nulas. 
    
\end{solution}

\begin{exercise}
    Mostre que uma matriz de rotação e uma matriz de reflexão são aplicações
    lineares ortogonais e, portanto, podem ser interpretadas como um movimento
    rígido.
\end{exercise}

\begin{solution}
    Uma matriz de rotação é uma transformação linear com parâmetro o ângulo
    $\theta$, de forma que 
    $$
    Rot = \begin{bmatrix}
        \cos(\theta) & -\sin(\theta) \\
        \sin(\theta) & \cos(\theta)
    \end{bmatrix}
    $$
    Observe que $\cos^2(\theta) + \sin^2(\theta) = 1$ e que
    $-\cos(\theta)\sin(\theta) + \sin(\theta)\cos(\theta) = 0$, portanto a
    rotação é uma aplicação linear ortogonal. De forma equivalente a reflexão
    é dada por 
    $$
    Ref = \begin{bmatrix}
        \cos(2\theta) & \sin(2\theta) \\
        \sin(2\theta) & -\cos(2\theta)
    \end{bmatrix}
    $$
    para chegar nessa matriz, observe que se queremos refletir um ponto por
    uma reta que faz ângulo $\theta$ com o eixo x, primeiro rotacionamos
    $-\theta$ para essa reta ficar no eixo x, refletimos em torno do eixo x e
    rotacionamos $\theta$ para voltar às coordenadas anteriores. A matriz
    $Ref$ tem a transposta (ela mesma) igual a inversa, 
    $$
    Ref^{-1} =  -1\begin{bmatrix}
        -\cos(2\theta) & -\sin(2\theta) \\
        -\sin(2\theta) & \cos(2\theta)
    \end{bmatrix}
    $$
    e, portanto, é ortogonal. Se considerarmos $p_0 = 0$, isto é, sem
    translação, ambos são movimentos rígidos. 
\end{solution}

\begin{exercise}
    Mostre que movimentos rígidos levam retas em retas e círculos em círculos.
\end{exercise}

\begin{solution}
    Considere um movimento rígido $\Phi$ como uma composição de uma transformação
    ortogonal $A$ e translação $p_0$. Tome três pontos $P, Q$ e $R$
    colineares. Defina $u = \vec{PQ}$ e $v = \vec{QR}$.  Como os pontos são
    colineares, $u$ e $v$ são paralelos (múltiplos). 
    
    Observe que $\Phi(Q) - \Phi(P) = A(Q - P) = Au$ e $\Phi(R) - \Phi(Q) =
    Av$. Assim, basta provar que $Au$ e $Av$ são paralelos e teremos que $\Phi(P),
    \Phi(Q)$ e $\Phi(R)$ são colineares, o que implica que um movimento rígido
    mantém colinearidade (na verdade uma transformação afim!). Como $u = kv
    \implies A(u) = kA(v)$ pela linearidade de $A$.  Concluí-se que $Au$ e
    $Av$ são paralelos e um movimento rígido transforma retas em retas. 

    \vspace{5mm}

    \noindent Agora vamos verificar que ele transforma círculos em círculos. Seja $C$ o centro do círculo e $r$ o raio. Tome um ponto $P$ qualquer do círculo. Por definição, 
    $$
    ||\Phi(P) - \Phi(C)|| = ||P - C|| = r
    $$
    e, portanto, $\Phi(P)$ está no círculo de centro $\Phi(C)$ e raio $r$.
    Isso prova que todo ponto do círculo de centro $C$ e raio $r$ reside no
    círculo de centro $\Phi(C)$ e raio $r$. Precisamos provar que a imagem
    dessa transformação é esse círculo (sobrejetividade). Assim, considere o
    círculo de centro $\Phi(C)$ e raio $r$. Tome um ponto $\hat{P}$ desse círculo e defina 
    $$P = \Phi^{-1}(\hat{P}) = A^{-1}(\hat{P} - p_0) = A^T(\hat{P} - p_0).$$
    Temos que $\Phi^{-1}$ é um movimento rígido (a transposta de uma matriz
    ortogonal é ortogonal) e, portanto 
    $$
    ||C - P|| = ||\Phi(C) - \hat{P}|| = r
    $$
    o que implica que $\Phi(P) = \hat{P}$ e o movimento rígido leva círculos
    em círculos. 

\end{solution}

\begin{exercise}
    Exemplifique, em ambiente computacional, movimentos rígidos sendo
    aplicados em uma curva de sua preferência.
\end{exercise}

\end{document}
