\documentclass[12pt,letterpaper]{article}

\usepackage[brazilian]{babel}
\usepackage[utf8]{inputenc}
\usepackage[T1]{fontenc}

\usepackage{fullpage}
\usepackage[top=2cm, bottom=4.5cm, left=2.5cm, right=2.5cm]{geometry}
\usepackage{amsmath,amsthm,amsfonts,amssymb,amscd}
\usepackage{lastpage}
\usepackage{enumerate}
\usepackage{fancyhdr}
\usepackage{mathrsfs}
\usepackage{xcolor}
\usepackage{graphicx}
\usepackage{listings}
\usepackage{hyperref}

\hypersetup{%
  colorlinks=true,
  linkcolor=blue,
  linkbordercolor={0 0 1}
}
 
\renewcommand\lstlistingname{Algorithm}
\renewcommand\lstlistlistingname{Algorithms}
\def\lstlistingautorefname{Alg.}

\lstdefinestyle{Python}{
    language        = Python,
    frame           = lines, 
    basicstyle      = \footnotesize,
    keywordstyle    = \color{blue},
    stringstyle     = \color{green},
    commentstyle    = \color{red}\ttfamily
}

\setlength{\parindent}{0.0in}
\setlength{\parskip}{0.05in}

% Edit these as appropriate
\newcommand\course{Lucas Moschen}
\newcommand\hwnumber{1}                  % <-- homework number
\newcommand\NetIDa{netid19823}           % <-- NetID of person #1
\newcommand\NetIDb{netid12038}           % <-- NetID of person #2 (Comment this line out for problem sets)

\pagestyle{fancyplain}
\headheight 35pt              % <-- Comment this line out for problem sets (make sure you are person #1)
\chead{\textbf{\Large Monitoria 7}}
\rhead{\course \\ \today}
\lfoot{}
\cfoot{}
\rfoot{\small\thepage}
\headsep 1.5em

\begin{document}

\section*{Definições e Teoremas}
\begin{itemize}
    \item Quando uma linha é substituída pela soma dela com um múltiplo de outra, a nova linha pertence ao mesmo subespaço gerado pelas primeiras. Mais do que isso, o subespaço gerado é o mesmo. (\textit{Michels})
    \item \textbf{Nulidade da Matriz: }Dimensão do espaço anulado da matriz $A$ ($anul(A)$). Você sabe encontrar a nulidade de uma matriz? \textbf{Teorema do Posto}! $anul(A) + posto(A) = n$.
    \item Considere $Ax = v_E$. Então, $A$ é a matriz de passagem da base $B$ para a base $E$, canônica. Assim, $v_B = A^{-1}v_E$. 
    \item Transformação Linear é uma função linear entre os espaços vetorias $E$ e $F$  com as propriedades de soma $T(u+v)=T(u)+T(v)$ e $T(\alpha u) = \alpha T(u)$.
    \item Lembre que $T(v) = T(\alpha_1 e_1 + ... + \alpha_n e_n) = \alpha_1 T(e_1) + ... + \alpha_n T(e_n)$. Logo, a transformação linear está definida quando conhcemos as imagens dos elementos de uma base. Daí saí a matriz de transformação.
    \item O escalonamento mantém a relação entre as colunas das matrizes. Para se ter a intuição, basta pensar que para resolver sistemas, escalonamos as matrizes, e as incógnitas permanecem as mesmas para o sistema escalonado.
    \item Suponha que temos uma vetor $w_\beta = (a,b)_\beta$ e queremos reescrever $w_E = (a',b')$, na base canônica. Para isso, precisamos fazer uma mudança de bases que envolve uma matriz de tranformação. Essa matriz é simples, pois é composta pelos vetores da base $\beta$. 
    \item \textbf{Teorema:} Seja uma transformação linear $A: E \to F$. A cada vetor $u \in \beta$ base de $E$, façamos corresponder um vetor $u' \in F$. Então essa tranformação, tal que $Au = u'$, é única. 
    \end{itemize}

\section*{Importante}
\begin{enumerate}
    \item Saber encontrar bases do espaço-coluna, do espaço-linha e do espaço-anulado (logo suas dimensões).
    \item Um funcional linear é $T: E \to \mathbb{R}$.
    \item Um operador linear é $T: E \to E$. 
    \item Lembrar de conjunto gerador. 
\end{enumerate}

\section*{Exercícios: }
\begin{enumerate}
    \item Prove que os seguintes polinômios são linearmente independentes: $p(x) = x^3 - 5x^2 + 1, q(x) = 2x^4 + 5x - 6, r(x) = x^2 - 5x + 2$. 
    \textit{Considere a base $X = \{1, x, x^2, x^3, x^4\}$} 
    \item Seja $X$ um conjunto de polinômios.  Se dois polinômios quaisquer de $X$ têm graus diferentes, $X$ é LI.
    \item Mostre que os vetores $u = (1,1)$ e $v = (-1,1)$ formam uma base de $\mathbb{R}^2$. 
    \item Avalie as afirmações: \\
    ( ) Seja $C = \{(x_1,x_2,x_3,x_4,x_5); x_i = 3\cdot x_{i-1}, i=2,...,5\}$. É um subespaço do $\mathbb{R}^5$. (Se verdadeiro, apresente uma base). \\
    ( ) É possível encontrar dois planos do $\mathbb{R}^4$ que se interesectem em apenas um ponto. (\textit{Pense em planos com dois parâmetos livres)}.\\
    ( ) O conjunto de todas as matrizes cujo determinante é maior do que zero é um subespaço das matrizes. \\
    ( ) A união de dois conjuntos LI é um conjunto LI, se  um deles é disjunto do subespaço gerado pelo outro. \\
    ( ) Existe apenas uma transformação linear com $T(0,0,1) = (1,2)$, $T(0,1,0) = (2,3), T(1,0,0) = (4,7) $, onde $T: \mathbb{R}^3\to \mathbb{R}^2$. Isto é, não existem $x,y,z$, tal que $T(x,y,z) \neq T'(x,y,z)$ com essas propriedades. \\
    (  ) Se $u, v, w \in E$ são colineares, então $Au,Av,Aw$ também são. \\
    (  ) Se $Aw = Au + Av$, então $w = u + v$. 
    \item Encontre os espaços linha, coluna e anulado da matriz:
    $$
    A = \left[
    \begin{array}{ccccc}
    -1 & 2 & 0 & 4 & 5\\
    3 & -7 & 2 & 0 & 1\\
    2 & -5 & 2 & 4 & 6
    \end{array}
    \right]
    $$
    \item Seja $U = \{u_1,u_2,u_3\}$. Como representar o vetor $(a,b,c)$, como combinação linear dos vetores de $U$. 
    \item Exiba uma base para o espaço vetorial formado pelos polinômios de grau $\leq n$ que se anulam em $x=2$ e $x=3$. Qual a dimensão dessa base? 
    \item Tem-se uma transformação linear $A(-1,1) = (1,2,3)$ e $A(2,3) = (1,1,1)$. Qual a matriz de tranformação de $A$, em relação às bases canônicas. 

    \end{enumerate}


\section*{Aplicação: Quadrados Mágicos}

Na semana passa, observamos a imagem da Melancolia I, de Albrecht Durer de 1514, em que apareceia um quadrado mágico clássico.

\textit{Para lembrar: definimos uma matriz $n\times n$ como quadrado mágico quando a soma de cada linha, coluna e diagonal é igual. Essa soma se chama peso. $Mag_n$ o conjunto de todos os quadrados mágicos de ordem $n$.O quadrado será clássico se usarmos os todos on números entre $1$ e $n^2$}. 

Considere $Mag_2$. Você conseguiria encontrar uma base para esse subespaço das matrizes 2 por 2 (provamos que é um subespaço na semana passada)? E $Mag_3$? 

\textbf{Exemplo: }$X = \{(1,1,1),(1,2,1)\}, Y = \{(1,0,0),(0,0,1)\}$

\end{document}
