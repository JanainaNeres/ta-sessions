\documentclass[a4paper,12pt]{article}

%%%%%%%%%%%%%%%%%%%%%%%%%%%%%%%%%%%%%%%%%%%%%%%%
% Packages
%%%%%%%%%%%%%%%%%%%%%%%%%%%%%%%%%%%%%%%%%%%%%%%%

\usepackage[right=2.5cm, left=2.5cm, top=2.5cm, bottom=2.5cm]{geometry} 
\usepackage[portuguese]{babel}
\usepackage[T1]{fontenc}
\usepackage[utf8]{inputenc}
\usepackage{enumerate}

% no indentation
%\usepackage{setspace}
%\setlength{\parindent}{0in}

\usepackage{graphicx} 
\usepackage{float}
\usepackage{xcolor}
\usepackage{tikz}
\usetikzlibrary{positioning}


\usepackage{mathtools}
\usepackage{amssymb, amsthm}

% headers
\usepackage{fancyhdr}

%%%%%%%%%%%%%%%%%%%%%%%%%%%%%%%%%%%%%%%%%%%%%%%%
% Proper definitions
%%%%%%%%%%%%%%%%%%%%%%%%%%%%%%%%%%%%%%%%%%%%%%%%
\newcommand{\R}{\mathbb{R}}
\newcommand{\Q}{\mathbb{Q}}
\newcommand{\Z}{\mathbb{Z}}
\newcommand{\tr}{\operatorname{tr}}
\newcommand\eq{\mathrel{\overset{\makebox[0pt]{\mbox{\normalfont\tiny\sffamily def}}}{=}}}

\newtheoremstyle{exer}{}{}{\color{blue}}{}{\color{blue}\bfseries}{}{ }{}

\newtheorem*{aff}{Afirmação}

\theoremstyle{exer}
\newtheorem{exercise}{Exercício}

\theoremstyle{definition}

\newcommand{\enu}[1]{\textcolor{blue}{#1}}

%%%%%%%%%%%%%%%%%%%%%%%%%%%%%%%%%%%%%%%%%%%%%%%%
% Header (and Footer)
%%%%%%%%%%%%%%%%%%%%%%%%%%%%%%%%%%%%%%%%%%%%%%%%

\pagestyle{fancy} 
\fancyhf{}

\lhead{\footnotesize ANC: Lista 1}
\rhead{\footnotesize Prof. Luciano} 
\cfoot{\footnotesize \thepage} 


\begin{document}

%%%%%%%%%%%%%%%%%%%%%%%%%%%%%%%%%%%%%%%%%%%%%%%%
% Title section of the document
%%%%%%%%%%%%%%%%%%%%%%%%%%%%%%%%%%%%%%%%%%%%%%%%

\thispagestyle{empty} 

\begin{tabular*}{0.95\textwidth}{l @{\extracolsep{\fill}} r} 
    {\large \bf Introdução à Análise Numérica 2021.2} &  \\
    Escola de Matemática Aplicada, Fundação Getulio Vargas &  \\
    Professor Hugo A. de la Cruz Cancino &  \\ 
    Monitor Lucas Machado Moschen & Entrega 20/08/2021\\
    \hline \\
\end{tabular*} 
\vspace*{0.3cm} 

\begin{center}
	{\Large \bf Lista 1} 
	\vspace{2mm}
	%{\bf Lucas Machado Moschen}	
\end{center}  
\vspace{0.4cm}

\begin{exercise}
    Determine a representação em ponto flutuante (considerando precisão dupla)
    do número $x = 20.1$. 
\end{exercise}

\begin{enumerate}[{\color{blue} 1.}]
    \item \enu{Qual o número de máquina de 64 bits usado para armazenar
    $fl(x)$ no computador? }

    \item \enu{Determine o valor exato do erro arredondado. Ou seja,
    determine: $20.1 - fl(20.1)$}
\end{enumerate}

\begin{exercise}
    Determine o equivalente decimal dos seguintes números de máquina em ponto
    flutuante.
\end{exercise}

\begin{enumerate}[{\color{blue} 1.}]
    \item \enu{\begin{tabular}{|c|c|c|}
            \hline
            1 & 10000001010
            &1001001100000000000000000000000000000000000000000000
            \\\hline
        \end{tabular}}

    \item \enu{\begin{tabular}{|c|c|c|}
            \hline
            0 & 01111111111
            & 0101001100000000000000000000000000000000000000000000
            \\\hline
        \end{tabular}}

    \item \enu{\begin{tabular}{|c|c|c|}
        \hline
        0 & 00000000000
        & 0101001100000000000000000000000000000000000000000000
        \\\hline
    \end{tabular}}

    \item \enu{\begin{tabular}{|c|c|c|}
        \hline
        1 & 11111111111
        & 0000000000000000000000000000000000000000000000000000
        \\\hline
    \end{tabular}}

    \item \enu{\begin{tabular}{|c|c|c|}
        \hline
        1 & 11111111111
        & 0000000000000000000000000000000000000000000000001111
        \\\hline
    \end{tabular}}

    \item \enu{Determine os próximos números de máquina para os números fornecidos
    nos itens anteriores, e escreva os mesmos na forma decimal.}

\end{enumerate}

\begin{exercise}
    Converte em binário ou converte em decimal, segundo seja o caso, e
    determine $fl(x)$: 
\end{exercise}

\begin{enumerate}[{\color{blue} 1. }]
    \item \enu{$x = 1/4$}
    \item \enu{$x = 1/3$}
    \item \enu{$x = 2/3$}
    \item \enu{$x = 0.9$}
    \item \enu{$x = 0.\bar{1000111}$}
    \item \enu{$x = 0.101\bar{100011}$}
\end{enumerate}

\begin{exercise}
    Para quais $k\in\mathbb{N}$ o número $5 + 2^k$ é representado de forma
    exata no computador. 
\end{exercise}

\begin{exercise}
    Considere a equação recursiva 
    \begin{equation}
        \label{eq:iteration}
        x_{n+1} = \frac{22}{7}x_n - \frac{3}{7}x_{n-1}; \qquad x_0 = 1, \quad x_1 = \frac{1}{7}
    \end{equation}
\end{exercise}

\begin{enumerate}[{\color{blue} 1.}]
    \item \enu{Demonstre que a equação acima tem solução $x_n =
    \left(\frac{1}{7}\right)^n$.} 
    \item \enu{Implemente o processo iterativo \eqref{eq:iteration} para calcular
    $x_n$.}
    \item \enu{Compare para diferentes valores de $n$ os valores de $x_n$ obtidos
    usando a solução em a) e usando a implementação computacional feita em b).
    Por que a partir de um certo valor de $n$ os valores são completamente
    diferentes? Reflita sobre isso!}

    \item \enu{Faça uma análise de estabilidade do algoritmo implementado em b)
    para calcular $x_n$.}

\end{enumerate}

\end{document}