\documentclass[a4paper,12pt]{article}

%%%%%%%%%%%%%%%%%%%%%%%%%%%%%%%%%%%%%%%%%%%%%%%%
% Packages
%%%%%%%%%%%%%%%%%%%%%%%%%%%%%%%%%%%%%%%%%%%%%%%%

\usepackage[right=2.5cm, left=2.5cm, top=2.5cm, bottom=2.5cm]{geometry} 
\usepackage[portuguese]{babel}
\usepackage[T1]{fontenc}
\usepackage[utf8]{inputenc}
\usepackage{enumerate}

% no indentation
%\usepackage{setspace}
%\setlength{\parindent}{0in}

\usepackage{graphicx} 
\usepackage{float}
\usepackage{xcolor}
\usepackage{tikz}
\usetikzlibrary{positioning}

\usepackage{url}
\usepackage{hyperref}

\usepackage{mathtools}
\usepackage{amssymb, amsthm}

% headers
\usepackage{fancyhdr}

%%%%%%%%%%%%%%%%%%%%%%%%%%%%%%%%%%%%%%%%%%%%%%%%
% Proper definitions
%%%%%%%%%%%%%%%%%%%%%%%%%%%%%%%%%%%%%%%%%%%%%%%%

\newcommand{\R}{\mathbb{R}}
\newcommand{\Q}{\mathbb{Q}}
\newcommand{\Z}{\mathbb{Z}}
\newcommand{\tr}{\operatorname{tr}}
\newcommand\eq{\mathrel{\overset{\makebox[0pt]{\mbox{\normalfont\tiny\sffamily def}}}{=}}}

\newtheoremstyle{exer}{}{}{\color{blue}}{}{\color{blue}\bfseries}{}{ }{}

\newtheorem*{aff}{Afirmação}

\theoremstyle{exer}
\newtheorem{exercise}{Exercício}

\theoremstyle{definition}

\newcommand{\enu}[1]{\textcolor{blue}{#1}}

%%%%%%%%%%%%%%%%%%%%%%%%%%%%%%%%%%%%%%%%%%%%%%%%
% Header (and Footer)
%%%%%%%%%%%%%%%%%%%%%%%%%%%%%%%%%%%%%%%%%%%%%%%%

\pagestyle{fancy} 
\fancyhf{}

\lhead{\footnotesize EDP: Lista 2}
\rhead{\footnotesize Prof. Moacyr} 
\cfoot{\footnotesize \thepage} 


\begin{document}

%%%%%%%%%%%%%%%%%%%%%%%%%%%%%%%%%%%%%%%%%%%%%%%%
% Title section of the document
%%%%%%%%%%%%%%%%%%%%%%%%%%%%%%%%%%%%%%%%%%%%%%%%

\thispagestyle{empty} 

\begin{tabular*}{0.95\textwidth}{l @{\extracolsep{\fill}} r} 
    {\large \bf Equações Diferenciais Parciais 2021.2} &  \\
    Escola de Matemática Aplicada, Fundação Getulio Vargas &  \\
    Professor Moacyr Alvim Horta Barbosa da Silva &  \\ 
    Monitor Lucas Machado Moschen & Entrega 02/09/2021\\
    \hline \\
\end{tabular*}
\vspace*{0.3cm} 

\begin{center}
	{\Large \bf Lista 2} 
	\vspace{2mm}
	%{\bf Lucas Machado Moschen}	
\end{center}  
\vspace{0.4cm}

\begin{exercise}
    Exiba as equações diferenciais características do problema de valor
    inicial 
    $$
    \begin{cases}
        F(u, u_t, u_x, x, t) = 0 \\
        u(x,0) = f(x),
    \end{cases}
    $$
    onde $u$ depende apenas de duas variáveis $(x, t)$.
\end{exercise}

\begin{exercise}
    Calcule as características das equações abaixo. Onde indicado, use as
    características para encontrar uma expressão da solução.
\end{exercise}

\begin{enumerate}[{\color{blue} (a) }]
    \item \enu{Advencção (transporte): 
    $$
    \begin{cases}
        u_t + c\cdot u_x = 0 \\
        u(x,0) = f(x).
    \end{cases}
    $$
    Encontre a solução para todo $x \in \R e t \ge 0$
    }

    \item \enu{
        Burgers:
        $$
        \begin{cases}
            u_t + c\cdot u\cdot u_x = 0 \\
            u(x,0) = f(x),
        \end{cases}
        $$
        com a condição inicial $$f(x) = \begin{cases}
            2, &x \le 0 \\
            -x/2 + 2, &x \in (0,2) \\
            1, &x \ge 2
        \end{cases}.
        $$
        Mostre que não há ocorrência de choques antes de $t = 2$. Faça um esboço de cinco momentos da solução integral,
        $h_i(x) = u(x, i)$ para $i=1,2,3,4$.}

        \item \enu{
            Exemplo livro Evans:
            $$
            \begin{cases}
                xu_t - t\cdot u_x = u, \\
                u(x,0) = f(x),
            \end{cases}
            $$
            $x \ge 0, t \ge 0$. Determine a solução $u(x, t)$ para todo $(x, t)$ no primeiro quadrante.   
        }
\end{enumerate}

\end{document}